\documentclass[12pt,a4paper]{article}
\usepackage[margin=1in,left=1.5in,top=1in,includefoot]{geometry}
\usepackage{graphicx}
\usepackage{float}
\usepackage{indentfirst}
\usepackage{pifont}
\usepackage{enumitem}
\usepackage{background}

\backgroundsetup{scale=1,angle=0,opacity=0.05,contents={\includegraphics[scale=2]{nsu.png}}}

\begin{document}
\NoBgThispage

\begin{titlepage}

  
  \begin{figure}[h!]
    \centering
    \includegraphics[scale=0.78]{nsu.png}
  \end{figure}
  
 \begin{center}
 \vspace{-0.3cm}
 \line(1,0){285}\\
 \Large{\bfseries NORTH SOUTH UNIVERSITY}\\
 
 \textsc{\large Department of Electrical and Computer Engineering}\\
 \vspace{1.4cm}
 \Large{\bfseries Project Proposal}\\
 \vspace{-0.3cm}
 \line(1,0){411}\\
 \Large{\bfseries RIDE SHARING COMPANY MANAGEMENT SYSTEM}\\
 [-0.4cm]  
  \line(1,0){175}\\
  \vspace{0.3cm}
   {\large\emph{ Database System Lab}}\\
  {\Large  \emph{\textbf{CSE311}}}\\
  {\large  \emph{Section:5}}\\
  {\large \emph{Summer 2021}}\\
 
 \vspace{0.85cm}
\begin{tabular}{p{8.5cm} p{10cm}}
\textsc{ \normalsize \textbf{Submitted By:}} & \textsc{\normalsize \textbf{Submitted To:}}\\

\textsc{ {\normalsize  Ahnaf Daiyan Azmir}} & \textsc{\normalsize \textbf{\textit{Course Faculty}}}\\

\textsc { \normalsize ID:1931027042} & \textsc{\normalsize Nadeem Ahmed}\\

\textsc{{ \normalsize Shadman Sakib}} & \textsc{\normalsize \textit{Senior Lecturer}}\\

\textsc{{ \normalsize ID:1931024042}} & \textsc{\normalsize \textbf{\textit{Lab Instructor}}}\\

\textsc{{ \normalsize Akil Hossain Prottoy}} & \textsc{\normalsize Nazmul Alam Dipto}\\

\textsc{{ \normalsize ID:1931795642}} & \\


\end{tabular}

\vspace{0.445cm}
\textit{{\small Date Of Submission: 24/07/2021}}

  
 \end{center}
 
 \end{titlepage}
 
 \newpage
 \,
 \section*{Introduction}
 In a country where the population is increasing at spiking rate, public transportation is something where the word comfort doesn’t fit. This is the reason popularity of ride sharing applications are rising and till date it has had a significant impact on the general public. Not only ride sharing transports provide a hassle-free environment but it also provides door-to-door convenience and safety. With such application not only drivers can make a living but they can also provide services to people. For people living in cities like Dhaka it is a kind of blessing to be able to book a ride while relaxing in their couch\
\vspace{-0.3cm}\begin{flushright}Although ride sharing transports like Shohoz, Uber might seem expensive but the service that we are here to provide will be the most economic and reliable choice for all general people. On the plus side our collection of vehicles is very rich.With fine quality of cars,we are here to provide the best service available.\end{flushright}
\,
\,
\section*{Objectives:}
\begin{itemize}
 \item To be able to book vehicles anytime anywhere.
 \item To provide a reliable, safe and economic means of transportation.
\item Not only for general purposes, our services are available for people who require heavy transportations like trucks/vans.
\item Creating job opportunities for people who are looking out to make a living (People can apply for a driving job easily through the website) 
\item People using this website can make some extra money by renting out their vehicles.
\item Get to know about drivers in service and condition of vehicles that are active in service.
\item Customers will be able to report an instant service feedback.
\item Admins will be able to look into databases of different departments existing in the company

\end{itemize}
\,
\newpage

\section*{Target Audience}
\begin{itemize}
\item For people who want a hassle-free transport service at a reasonable pricing
\item For people who don’t own a car and does not want to bear the maintenance of a vehicle 
\item People who are looking for a full-time/part-time job and are willing to earn money through driving
\end{itemize}
\,
\section*{Value Proposition}
\begin{flushleft}
This website will reduce the struggle for people who have to go out and look for a taxi or other sorts of rental vehicles. 
With a more friendly user interface anyone can easily use our website and can easily book themselves a ride.
\end{flushleft}
\begin{flushright}
Our website is open for customers to look at cars that are in service and they are also welcome to fill out forms to give out their vehicles for rental. 
Our services are cheaper than conventional taxis.
\end{flushright}
\begin{flushleft}
Users can immediately report by calling at our helpline if they face any inconvenience caused by our service/service provider. Our helpline is open 24/7 and immediate actions are applied.
\end{flushleft}
\,
\section*{{Resources that will be used to create the website:}}
\begin{enumerate}
\item {\large \textbf{HTML}}, \textit{HTML will be used to make the page layout}
\item {\large \textbf{CSS}}, \textit{CSS will be used for the designing part}
\item {\large \textbf{PHP}}, \textit{PHP will be used to implement all the front-end logic}
\item {\large \textbf{MYSQL}}, \textit{MYSQL will be used as the database}
\item {\large \textbf{Web Server}}
\item {\large \textbf{JavaScript}}, \textit{JavaScript will be used to create small amount of animations in the website layout and will also be used for validation tasks}
\end{enumerate}

\newpage
 \,
\section*{\textbf{{\Large Features of the web page with description:}}}
\begin{flushleft}
Web page will open with a general view and there will be options in the navigation bar for the user to choose from. Anyone having the access of the domain will be able to view it.
\end{flushleft}
\begin{flushright}
The homepage will be having a good and user-friendly interface, there will be an “About Us” page which will provide a brief description about the whole project. The webpage will also contain “Contact Us” which will allow the users to contact the company for any sort of queries.
\end{flushright}
\begin{flushleft}
There will admin panel for the employees and other respective panels for the users of the website.
\end{flushleft}
\vspace{0.5cm}
\textbf{Functionalities for the Admin:}

\begin{itemize}[label=\ding{212}]

\item Manage vehicle database \begin{itemize} \item Add new vehicles to database \item Remove vehicles from database \item View Details of the vehicles \end{itemize}

\item Manage customers and their bookings (View details of users using the website)

\item Manage the Drivers database \begin{itemize} \item Add drivers, Remove drivers \item View Details of the Drivers \end{itemize}

\item Manage other departmental databases
\end{itemize}
\vspace{0.3cm}

\textbf{Functionalities for the Users:}
\begin{itemize}[label=\ding{212}]
\item Users will be able to register their account on the website
\item Users will login in and book a ride 
\item Users will be able to choose from a class of vehicles (Economy, Premium, Truck, Minivans)
\item Users can use the report button to report any inconveniences caused by the company service and get immediate help by simply putting out their phone number. A customer service representative will be there assisting the user.
\item Users can view all the vehicles that are currently active.
\item Users can make an appointment for renting out their cars by simply filling out a form
\item Anyone can apply for a job position (if available) by filling out a form
\item There will be a feedback system where users can share their travel experience and rate the service
\end{itemize}
 \,
\section*{Challenges}
As it is a completely digitalized platform many users/drivers might find it difficult to adapt, thus we have to make sure to make the user interface as simple as we can so that normal people can access and use it to book rides. Connecting the databases and making sure all the individual elements work properly is another challenge that we have to overcome. Other challenges come with the reliability of the data and verification of the data. Checking the reliability of the data is a very convoluted task. As we know that the competition for ride sharing companies is high hence we have to design our operations in such way so that users of the website find our service easy and accessible.
\vspace{11cm}
\begin{center}
 \textit{{\small Thank you}}
\end{center} 


\end{document}
